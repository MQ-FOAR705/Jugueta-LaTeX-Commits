\documentclass{article}
\usepackage[utf8]{inputenc}
\usepackage{graphicx}
\setlength{\parindent}{0pt}
\setlength{\parskip}{1em}

\title{FOAR705 - Learning Journal Week 7 - OpenRefine}
\author{Jan Jugueta - 44828020}
\date{September 2019}

\begin{document}

\maketitle

\tableofcontents

\newpage
\section{OpenRefine Lessons}

\subsection{Creating a new OpenRefine project}

\textbf{9/9/19 - 2:01pm}

\textbf{Objective:} Create a new project

\textbf{Action:}

\begin{enumerate}
    \item Clicked on create project
    \item Selected get data from 'This Computer'
    \item Clicked on Choose Files
    \item Selected SAFI\_openrefine.csv
    \item Clicked Next
    \item Clicked Create Project
\end{enumerate}

\textbf{Error:} None.

\textbf{Result:} New project created with OpenRefine using SAFI\_openrefine.csv.

\subsection{Using Facets}

\textbf{9/9/19 - 2:11pm}

\textbf{Objective:} Use faceting to look for potential errors in data entry in the village column.

\textbf{Action:}

\begin{enumerate}
    \item Selected Text Facet in the Facet in the drop down menu from the village column
\end{enumerate}

\textbf{Error:} None.

\textbf{Result:} The facet window on the left shows all the data entries for the village column. With this view we can see the errors that have been made. Chirdozo is likely a typo. Ruca is like a typo as well. There are many entries for Ruaca - Nhamuenda. 49 is almost certainly an error, but I have no idea what it refers to so there's not much I can do about that one.

\subsection{Using clustering to detect possible typing errors}

\textbf{9/9/19 - 2:35pm}

\textbf{Objective:} To merge clusters to clean up the values in the village column

\textbf{Action:}

\begin{enumerate}
    \item Click Cluster in the village Text Facet
    \item Select the key collision method and the metaphone3 keying function
    \item Click the Merge? box beside the clusters
    \item Click Merge Selected and Recluster
    \item Tried all other Methods and Keying Functions
    \item Go back to the village Text Facet
    \item Hovered over Chirdozo and selected edit
    \item Renamed Chirdozo to Chirodzo
    \item Hovered over Ruaca-Nhamuenda and selected edit
    \item Renamed it to Ruaca.
\end{enumerate}

\textbf{Error:} None.

\textbf{Result:} There are now only 4 different values in the village column. Chirodzo, God, Ruaca and 49. Another point was that the other methods and keying functions did not find any more clusters.

\subsection{Transforming data}

\textbf{9/9/19 - 2:46pm}

\textbf{Objective:} Remove the left square brackets from the data in the items\_owned column.

\textbf{Action:}

\begin{enumerate}
    \item Clicked on the arrow at the top of the items\_owned column
    \item Select Edit Cells
    \item Select Transform...
    \item Type in \begin{verbatim}
        value.replace("[", "")]
    \end{verbatim}
    \item Click OK
\end{enumerate}

\textbf{Error:} None.

\textbf{Result:} All left brackets have been removed.

\subsection{Trim Leading and Trailing Whitespace}

\textbf{9/9/19 - 3:12pm}

\textbf{Objective:} Tidy up respondent\_wall\_type so that burntbricks and muddaub are just one value instead of two.

\textbf{Action:}

\begin{enumerate}
    \item Selected Edit cells
    \item Selected Common transforms
    \item Selected Trim leading and trailing whitespace
\end{enumerate}

\textbf{Error:} None.

\textbf{Result:} There are only four choices left in the text facet.

\subsection{Filtering}

\textbf{10/9/19 - 4:48pm}



\newpage
\section{OpenRefine Exercises}

\subsection{Using Facets}

\textbf{9/9/19 - 2:17pm}

\begin{enumerate}
    \item There are 19 different interview\_date values
    \item It appears as it is formatted by Text.
    \item \textbf{Objective:} Change format to Date.
    
    \textbf{Action:}
    \begin{enumerate}
        \item Clicked the drop down menu in the interview\_date column
        \item Selected Edit cells
        \item Selected Common transforms
        \item Selected To date
    \end{enumerate}
    
    \textbf{Error:} None
    
    \textbf{Result:} Data format has been changed to date. 2016-11-16T00:00:00Z is an example of the format.
    \item Most of the data was collected in November 2016.
\end{enumerate}

\subsection{Transforming data}

\textbf{9/9/19 - 2:50pm}

\textbf{Objective:} Remove the right square brackets, single quote marks and spaces from the data in the items\_owned column.

\textbf{Action:}

\begin{enumerate}
    \item Clicked on the arrow at the top of the items\_owned column
    \item Select Edit Cells
    \item Select Transform...
    \item Type in \begin{verbatim}
        value.replace("'", "")
        value.replace("]", "")
        value.replace(" ", "")
    \end{verbatim}
    \item Click OK
\end{enumerate}

\textbf{Error:} None.

\textbf{Result:} All right brackets, single quote marks and spaces have been removed.

\textbf{9/9/19 - 2:54pm}

By sorting with count we can see which are the most commonly owned items. They are the mobile phone and radio. The least commonly owned items are cars and computers.

\textbf{9/9/19 - 2:57pm}

\textbf{Objective:} Finding which month(s) farmers were more likely to lack food

\textbf{Action:}

\begin{enumerate}
    \item Select Transform... in the Edit cells drop down menu in the months\_lack\_food column.
    \item Enter expression \begin{verbatim}
        value.replace("[", "").replace("]", "").replace(" ", "").replace("'", "")
    \end{verbatim}
    \item Create Custom text facet for months\_lack\_food column.
    \item Enter expression \begin{verbatim}
        value.split(";")
    \end{verbatim}
    \item Sort by count
\end{enumerate}

\textbf{Error:} None.

\textbf{Result:} November was the most common month that farmers lacked food.

\textbf{9/9/19 - 3:04pm}

\textbf{Objective:} Clean up months\_no\_water, liv\_owned, res\_change, and no\_food\_mitigation columns.

\textbf{Action:}

\begin{enumerate}
    \item Clicked on months\_no\_water column
    \item Selected Transform...
    \item Went to the History tab
    \item Reused the last expression in history
    \item Repeated the process for liv\_owned, res\_change, and no\_food\_mitigation columns
\end{enumerate}

\textbf{Error:} None

\textbf{Result:} Cleaned up months\_no\_water, liv\_owned, res\_change, and no\_food\_mitigation columns.

\subsection{Using undo and redo}

\textbf{9/9/19 - 3:09pm}

The undo and redo functions work just like explained in the lesson.

\newpage
\section{OpenRefine Lesson Notes}

\begin{itemize}
    \item Data is often very messy. OpenRefine can help organise messy data.
    \item OpenRefine records all changes to the data.
    \item OpenRefine does not modify the original dataset.
    \item OpenRefine is open source.
    \item Facets help get an overview of the data as well as bring consistency to the data.
\end{itemize}

\end{document}
